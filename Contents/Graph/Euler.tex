七橋問題根據起點與終點是否相同,分成 Euler path(不同)及 Euler circuit(相同)。

\begin{itemize}
    \item 判斷法
    \item 無向圖部分,將點分成奇點(度數為奇數)和偶點(度數為偶數)。
    \begin{itemize}
        \item Euler path:奇點數為 0 或 2
        \item Euler circuit:沒有奇點
    \end{itemize}
    \item 有向圖部分,將點分成出點(出度 - 入度 = 1)和入點(入度 - 出度 = 1)還有平衡點(出度 = 入度)。
    \begin{itemize}
        \item Euler path:出點和入點個數同時為 0 或 1。
        \item Euler circuit:只有平衡點。
    \end{itemize}
    \item 求出一組解
    \item 用 DFS 遍歷整張圖,設 S 為離開的順序,無向圖的答案為 S ,有向圖的答案為反向的 S 。
    \item DFS 起點選定:
    \begin{itemize}
        \item Euler path:無向圖選擇任意一個奇點,有向圖選擇出點。
        \item Euler circuit:任意一點。
    \end{itemize}
\end{itemize}

Code from Eric

\begin{lstlisting}
#define ll long long
#define PB push_back
#define EB emplace_back
#define PII pair<int, int>
#define MP make_pair
#define all(x) x.begin(), x.end()
#define maxn 50000+5
  
//structure
struct Eular {
  vector<PII> adj[maxn];
  vector<bool> edges;
  vector<PII> path;
  int chk[maxn];
  int n;

  void init(int _n) {
    n = _n;
    for (int i = 0; i <= n; i++) adj[i].clear();
    edges.clear();
    path.clear();
    memset(chk, 0, sizeof(chk));
  }

  void dfs(int v) {
    for (auto i : adj[v]) {
      if (edges[i.first] == true) {
        edges[i.first] = false;
        dfs(i.second);
        path.EB(MP(i.second, v));
      }
    }
  }

  void add_Edge(int from, int to) {
    edges.PB(true);

    // for bi-directed graph
    adj[from].PB(MP(edges.size() - 1, to));
    adj[to].PB(MP(edges.size() - 1, from));
    chk[from]++;
    chk[to]++;

    // for directed graph
    // adj[from].PB(MP(edges.size()-1, to));
    // check[from]++;
  }

  bool eular_path() {
    int st = -1;
    for (int i = 1; i <= n; i++) {
      if (chk[i] % 2 == 1) {
        st = i;
        break;
      }
    }
    if (st == -1) {
      return false;
    }
    dfs(st);
    return true;
  }

  void print_path(void) {
    for (auto i : path) {
      printf("%d %d\n", i.first, i.second);
    }
  }
};
\end{lstlisting}

Code from allen(lexicographic order)

\begin{lstlisting}
#include <bits/stdc++.h>
using namespace std;
const int ALP = 30;
const int MXN = 1005;
int n;
int din[ALP], dout[ALP];
int par[ALP];
vector<string> vs[MXN], ans;
bitset<MXN> vis, used[ALP];

void djsInit() {
  for (int i = 0; i != ALP; ++i) {
    par[i] = i;
  }
}

int Find(int x) { return (x == par[x] ? (x) : (par[x] = Find(par[x]))); }

void init() {
  djsInit();
  memset(din, 0, sizeof(din));
  memset(dout, 0, sizeof(dout));
  vis.reset();
  for (int i = 0; i != ALP; ++i) {
    vs[i].clear();
    used[i].reset();
  }
  return;
}

void dfs(int u) {
  for (int i = 0; i != (int)vs[u].size(); ++i) {
    if (used[u][i]) {
      continue;
    }
    used[u][i] = 1;
    string s = vs[u][i];
    int v = s[s.size() - 1] - 'a';
    dfs(v);
    ans.push_back(s);
  }
}

bool solve() {
  int cnt = 1;
  for (int i = 0; i != n; ++i) {
    string s;
    cin >> s;
    int from = s[0] - 'a', to = s.back() - 'a';
    ++din[to];
    ++dout[from];
    vs[from].push_back(s);
    vis[from] = vis[to] = true;
    if ((from = Find(from)) != (to = Find(to))) {
      par[from] = to;
      ++cnt;
    }
  }
  if ((int)vis.count() != cnt) {
    return false;
  }
  int root, st, pin = 0, pout = 0;
  for (int i = ALP - 1; i >= 0; --i) {
    sort(vs[i].begin(), vs[i].end());
    if (vs[i].size()) root = i;
    int d = dout[i] - din[i];
    if (d == 1) {
      ++pout;
      st = i;
    } else if (d == -1) {
      ++pin;
    } else if (d != 0) {
      return false;
    }
  }
  if (pin != pout || pin > 1) {
    return false;
  }
  ans.clear();
  dfs((pin ? st : root));
  return true;
}

int main() {
  int t;
  cin >> t;
  while (t--) {
    cin >> n;
    init();
    if (!solve()) {
      cout << "***\n";
      continue;
    }
    for (int i = ans.size() - 1; i >= 0; --i) {
      cout << ans[i] << ".\n"[i == 0];
    }
  }
}
\end{lstlisting}